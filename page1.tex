%%%%%%%%%%%%%%%%%% file template.tex %%%%%%%%%%%%%%%%%%%%%%%%%
%
% This is a general template file for the LaTeX package SVJour3
% for Springer journals.          Springer Heidelberg 2010/09/16
%
% Copy it to a new file with a new name and use it as the basis
% for your article. Delete % signs as needed.
%
% This template includes a few options for different layouts and
% content for various journals. Please consult a previous issue of
% your journal as needed.
%
%%%%%%%%%%%%%%%%%%%%%%%%%%%%%%%%%%%%%%%%%%%%%%%%%%%%%%%%%%%%%%%%%%%
%
% First comes an example EPS file -- just ignore it and
% proceed on the \documentclass line
% your LaTeX will extract the file if required

%\documentclass{svjour3}                     % onecolumn (standard format)
%\documentclass[smallcondensed]{svjour3}     % onecolumn (ditto)
\documentclass[smallextended]{svjour3}       % onecolumn (second format)
%\documentclass[twocolumn]{svjour3}          % twocolumn
%
\smartqed  % flush right qed marks, e.g. at end of proof
%
\usepackage{graphicx}
\usepackage{multirow}
\usepackage{alltt}

\usepackage{amsmath} % Required for \DeclareMathOperator*
\DeclareMathOperator*{\argmax}{arg\,max}
\usepackage{amssymb,amsfonts}
\usepackage{colortbl}
\usepackage{pifont}
\usepackage{graphicx}
\usepackage{textcomp}
\usepackage{booktabs}
\usepackage{algorithm}
\usepackage{algpseudocode}
\usepackage{fancyvrb}
\usepackage{multirow}
\usepackage[table,xcdraw]{xcolor}
\usepackage[utf8]{inputenc}
\usepackage{fancyhdr}
%\usepackage{lmodern}
\usepackage{url}
% \usepackage[hidelinks]{hyperref}
\usepackage{adjustbox}

\usepackage{wrapfig} 
\usepackage{changepage} 
\usepackage{framed}
\usepackage{enumitem}
\usepackage{cite}

\usepackage{graphicx}
\usepackage{caption}
\usepackage{subcaption}
\usepackage{longtable}

\definecolor{niceblue}{HTML}{0000FF}

\newcommand{\BLUE}{\color{niceblue}}
\newcommand{\BLACK}{\color{black}}

 \usepackage[hidelinks]{hyperref}

\newcommand{\there}[1]{%
  \hyperlink{resp:#1}{%
    %\textcolor{gray}{➜}~%
    \fcolorbox{black}{black!15}{%
      \bfseries\scriptsize #1%
    }~on page \pageref{resp:#1}%
  }%
}


\newcommand{\here}[1]{%
  \hypertarget{resp:#1}{}% create the target anchor
  \fcolorbox{black}{black!15}{%
    \label{resp:#1}%
    \bfseries\scriptsize  {#1}%
  }%
}



\definecolor{lightblue}{rgb}{0.85,0.9,1}
\definecolor{blue}{rgb}{0.5,0.75,1}
\definecolor{darkblue}{rgb}{0,0.5,1}

\definecolor{lightred}{rgb}{1,0.85,0.85}
\definecolor{red}{rgb}{1,0.5,0.5}
\definecolor{darkred}{rgb}{1,0,0}

\definecolor{codeblue}{rgb}{0.1,0.1,0.7}
\definecolor{codegreen}{rgb}{0.1,0.6,0.1}
\definecolor{codegray}{rgb}{0.5,0.5,0.5}
\definecolor{codepurple}{rgb}{0.58,0,0.82}
\definecolor{backcolour}{rgb}{0.95,0.95,0.92}
\definecolor{verylightgray}{rgb}{0.95, 0.95, 0.95}

\newcommand{\coloredcell}[2]{\cellcolor{#1}{\rule{0pt}{0.5ex}\scriptsize #2}}

\usepackage{bm}
%
% \usepackage{mathptmx}      % use Times fonts if available on your TeX system
%
% insert here the call for the packages your document requires
%\usepackage{latexsym}
% etc.
%
% please place your own definitions here and don't use \def but
% \newcommand{}{}
%
% Insert the name of "your journal" with
% \journalname{myjournal}
%
\begin{document}
\newcommand{\bi}{\begin{itemize}}
\newcommand{\ei}{\end{itemize}}
\title{From Brittle to Robust: \\Improving LLM Annotations for SE Optimization 
%If Not One Then Many: Ensemble of Few Shot Learners for Uncommon SE Optimization Problems %\thanks{Grants or other notes
%about the article that should go on the front page should be
%placed here. General acknowledgments should be placed at the end of the article.}
}


%\titlerunning{Short form of title}        % if too long for running head

\author{Lohith Senthilkumar         \and
        Tim Menzies %etc.
}

%\authorrunning{Short form of author list} % if too long for running head

\institute{All authors are from Computer Science, North Carolina State University
   Oval Dr,  Raleigh, NC 27606
\\\email{panjal@ncsu.edu, timm@ieee.org}}

\date{Received: date / Accepted: date}
% The correct dates will be entered by the editor

 
\maketitle

\begin{abstract}
Software analytics often builds 
from labeled data.
Labeling   can be slow, error prone, and expensive.
When human expertise is scarce,
SE researchers sometimes
ask large language models (LLMs) for the   missing labels.

While this has been successful in some domains,
recent results show that LLM-based  labeling has blind spots.
Specifically, their labeling is not effective for  higher dimensional multi-objective problems.


To address this task, we propose a novel LLM prompting strategy called SynthCore. When one opinion fails,  SynthCore's   combines multiple separated opinions generated by LLMs (with no knowledge of each others' answers) into
an ensemble of few-shot learners. 
Simpler than other strategies (e.g. chain-of-thought, multi-agent-debate, etc)   SynthCore   aggregates results from    multiple single prompt sessions (with no crossover between them). 

  SynthCore has been tested on 49 SE multi-objective optimization tasks,
 handling tasks as diverse as software project management, Makefile configuration, and hyperparameter optimization.
SynthCore's
ensemble   found optimizations
that are better than state-of-the-art alternative approaches  (Gaussian Process Models, Tree of Parzen Estimators,
active learners in both exploration and exploitation mode). 
Importantly, these optimizations were made using data labeled by LLMs, without any human opinions.

From these experiments, we conclude that ensembles of few shot learners can successfully annotate high dimensional multi-objective tasks. Further, we speculate that other successful few-shot prompting results could be quickly and easily enhanced  using SynthCore's ensemble approach.

To support open science, all our data and scripts are available  at \url{https://github.com/lohithsowmiyan/lazy-llm/tree/clusters}.


 

\keywords{Active Learning \and Large Language Models \and Multi-Objective Optimization}
% \PACS{PACS code1 \and PACS code2 \and more}
\end{abstract}



\end{document}
